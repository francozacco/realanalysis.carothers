\documentclass[11pt]{article}
\usepackage{amssymb}
\usepackage{amsthm}
\usepackage{enumitem}
\usepackage{amsmath}
\usepackage{bm}
\usepackage{adjustbox}
\usepackage{mathrsfs}
\usepackage{graphicx}
\usepackage{siunitx}
\usepackage[mathscr]{euscript}

\title{\textbf{Solved selected problems of Real Analysis - Carothers}}
\author{Franco Zacco}
\date{}

\addtolength{\topmargin}{-3cm}
\addtolength{\textheight}{3cm}

\newcommand{\N}{\mathbb{N}}
\newcommand{\Z}{\mathbb{Z}}
\newcommand{\Q}{\mathbb{Q}}
\newcommand{\R}{\mathbb{R}}

\theoremstyle{definition}
\newtheorem*{solution*}{Solution}

\begin{document}
\maketitle
\thispagestyle{empty}

\section*{Chapter 1 - Calculus Review}

	\begin{proof}{\textbf{1}}
        Let us take the set $-A = \{-a ~|~ a \in A\}$ since $A$ is bounded below then
        $-A$ is bounded above and because of the The Least Upper Bound Axiom we know
        there is $u = \sup(-A)$ such that if $x \in -A$ then $x \leq \sup(-A)$
        so $-x \geq -\sup(-A)$ and $-x \in A$ then $-\sup(-A)$ is an lower bound of $A$.

        Now let us suppose that $l$ is a Lower Bound of $A$ such that 
        $-x \geq l > -\sup(-A)$ we want to
        arrive to a contradiction to show there is no such an $l$. Then
        $x \leq -l < \sup(-A)$ but $\sup(-A)$ is the Least Upper Bound of $-A$ and this
        cannot be so we have a contradiction and must be the case that if $l$ is a Lower
        Bound then $l \leq -\sup(-A)$.

        Therefore $-\sup(-A)$ is the Greatest Lower Bound of $A$.  
    \end{proof}
\cleardoublepage
    \begin{proof}{\textbf{3}} Supremum characterization.\\
        ($\rightarrow$)
        \begin{itemize}
            \item [(i)] If $s = \sup(A)$ then $s$ is the Least Upper Bound
            for $A$ so by definition $s$ is an upper bound for $A$.
            \item[(ii)] Let $\epsilon>0$. Since $s$ is the Least Upper Bound
            of $A$, then $s-\epsilon<s$ and $s - \epsilon$ cannot be an upper bound of
            $A$ thus there exists $a \in A$ such that $s \geq a > s - \epsilon$.
        \end{itemize}
        ($\leftarrow$)
        Now we want to show by contradiction that $s$ is the Least Upper Bound for $A$. 
        Suppose $u \neq s$ is the Least Upper Bound for $A$ so $u = \sup(A)$ which means that
        if $a \in A$ then $a \leq u$ and since $s$ is an upper bound for $A$ then
        $u<s$. We also have that $a>s-\epsilon$ for every $\epsilon > 0$ so let us take
        $\epsilon = s - u$ then we have that $a>s - (s-u) = u$ but we said that
        $a \leq u$ which means that we have a contradiction. Therefore $s$ must be the
        Least Upper Bound for $A$ i.e. $s = \sup(A)$.

        Infimum characterization. Let $A$ be a nonempty set of $\R$ that is bounded
        below. We want to prove that $i = \inf(A)$ if and only if (i) $i$ is a lower
        bound for $A$, and (ii) for every $\epsilon >0$ there is an $a \in A$ such that
        $a< i + \epsilon$.\\
        ($\rightarrow$)
        \begin{itemize}
            \item [(i)] If $i = \inf(A)$ then $i$ is the Greatest Lower Bound
            for $A$ so by definition $i$ is a lower bound for $A$.
            \item[(ii)] Let $\epsilon>0$. Since $i$ is the Greatest Lower Bound
            for $A$, then $i+\epsilon>i$ and $i + \epsilon$ cannot be a lower bound of
            $A$ thus there exists $a \in A$ such that $i \leq a < i + \epsilon$.
        \end{itemize}
        ($\leftarrow$)
        Now we want to show by contradiction that $i$ is the Greatest Lower Bound for $A$. 
        Suppose $l \neq i$ is the Greatest Lower Bound for $A$ so $l = \inf(A)$ which
        means that if $a \in A$ then $a \geq l$ and since $i$ is an lower bound for $A$
        then $i<l$. We also have that $a<i+\epsilon$ for every $\epsilon > 0$ so let us
        take $\epsilon = l-i$ then we have that $a<i + (l-i) = l$ but we said that
        $a \geq l$ which means that we have a contradiction. Therefore $i$ must be the
        Least Upper Bound for $A$ i.e. $i = \inf(A)$.
    \end{proof}
\cleardoublepage
    \begin{proof}{\textbf{6}}
        Let the sequence $(a_n)$ to be  convergent to $a \in \R$, so for every positive
        $\epsilon >0$, there is $N \in \N$ such that $|a_n - a| < \epsilon$ whenever
        $n \geq N$. Also let us notice that
        $$|a_n| = |a_n - a + a| \leq |a_n - a| + |a| < \epsilon + |a|$$
        so in summary $|a_n| < |a| + \epsilon$. Let us take then
        $$M = max\{|a_1|, |a_2|, ..., |a_n|, |a| + \epsilon\}$$
        so we see that $|a_n| < M$ and therefore $(a_n)$ is bounded.

        Given that $(a_n)$ is bounded below and above then because of The Least Upper
        Bound Axiom and The Greatest Lower Bound Axiom we know that $(a_n)$ has a
        Supremum and an Infimum.

        Now we want to show by contradiction that $a\leq \sup(a_n)$.
        Let us suppose that $\sup(a_n) < a$ then if we take $\epsilon = a - \sup(a_n) > 0$
        we have that there must be some $N \in \N$ such that if $n \geq N$ then 
        $|a_n - a| < a - \sup(a_n)$ so this means that
        $-a + \sup(a_n) < a_n - a < a - \sup(a_n)$ then 
        $\sup(a_n)< a_n < 2a - \sup(a_n)$ but $\sup(a_n)$ is the supremum of $a_n$ so
        we have a contradiction. Therefore must be the case that $a\leq \sup(a_n)$.

        In the same way we want to show by contradiction that $\inf(a_n) \leq a$.
        Let us suppose now that $\inf(a_n) > a$ then if we take
        $\epsilon = \inf(a_n) - a > 0$
        we have that there must be some $N \in \N$ such that if $n \geq N$ then 
        $|a_n - a| < \inf(a_n) - a$ so this means that
        $a-\inf(a_n) < a_n - a < \inf(a_n) - a$ then 
        $2a - \inf(a_n)< a_n < \inf(a_n)$ but $\inf(a_n)$ is the Infimum of $a_n$ so
        we have a contradiction. Therefore must be the case that $\inf(a_n) \leq a$.
    \end{proof}
    \begin{proof}{\textbf{7}}
        Since $b-a>0$ we can apply Lemma 1.2 to get a positive integer $q'$ such that
        $q'(b-a)>1$ we also know that $\sqrt{2}>1$ so $\sqrt{2}q'(b-a)>1$ then we see that
        $\sqrt{2}q'b$ is bigger than $\sqrt{2}q'a$ by a value bigger than 1 so this means
        that there is some $p \in \Z$ between them, thus $\sqrt{2}q'b > p > \sqrt{2}q'a$
        then it follows that $a < \sqrt{2}p/q < b$ where we used that $2q' = q$
        and that $\sqrt{2}\cdot\sqrt{2} = 2$. Therefore there is some irrational number
        of the form $\sqrt{2}p/q$ between $a$ and $b$.
    \end{proof}
\cleardoublepage
    \begin{proof}{\textbf{13}}

        ($\rightarrow$) We know that $(s_n)$ converges so let $\epsilon > 0$ it follows
        then that there is some $s \in \R$ such that when $n\geq N$ then
        $|s_n - s|< \epsilon$ which means that $|s_n| < |s| + \epsilon$.
        Let us take then $M = \max\{|s_1|, |s_2|, ..., |s_n|, |s|+\epsilon \}$, so we 
        see that $|s_n| < M$ and since $a_n \geq 0$ then
        $s_n = \sum_{i=1}^{n} a_i \geq 0$ which means that $s_n < M$.
        Therefore $(s_n)$ is bounded.
        
        ($\leftarrow$) Since we know now that $(s_n)$ is bounded we want to prove by
        induction that it's a monotone (increasing) sequence.
        First we see that $a_1 \geq 0$ and $a_2 \geq 0$ then
        $s_1 = a_1 \leq a_1 + a_2 = s_2$.\\
        Now let us suppose that the following expression is true
        $$s_{n-1} = \sum_{i=1}^{n-1} a_i \leq \sum_{i=1}^{n} a_i = s_n$$
        then since $a_{n+1} \geq 0$ we have that 
        $$s_{n} = \sum_{i=1}^{n} a_i \leq \sum_{i=1}^{n} a_i + a_{n+1} = s_{n+1}$$
        Therefore we showed that $(s_n)$ is bounded and monotone it follows then that
        it is convergent.
    \end{proof}
    \begin{proof}{\textbf{22}}
        Let us prove first by contradiction that $\inf_n a_n \leq \lim \inf_{n \to \infty} a_n$.
        Suppose $\inf_n a_n > \lim \inf_{n \to \infty} a_n = \sup t_n$ then
        $\inf_n a_n > \sup t_n \geq t_n$ but we know that $\inf_n a_n \leq t_n$ so we
        have a contradiction then it must be the case that
        $\inf_n a_n \leq \lim \inf_{n \to \infty} a_n$.

        Now let us prove by contradiction that $\lim \sup_{n \to \infty} a_n \leq \sup_n a_n$.
        Suppose $\inf T_n = \lim \sup_{n \to \infty} > \sup_n a_n$ then
        $T_n \geq \inf T_n >\sup_n a_n$ but we know that $T_n \leq \sup_n a_n$ so we
        have a contradiction then it must be the case that
        $\lim \sup_{n \to \infty} a_n \leq \sup_n a_n$.

        Finally we want to prove that
        $$\sup t_n =\lim \inf_{n \to \infty} a_n \leq \lim \sup_{n \to \infty} a_n = \inf T_n$$
        We know that $t_n \leq T_n$ so if we take limits in both sides we have that
        $\lim_{n \to \infty} t_n \leq \lim_{n \to \infty}T_n$ and since $a_n$ is
        bounded then we have that
        $$\lim \inf_{n \to \infty} a_n = \lim_{n \to \infty} t_n \leq 
        \lim_{n \to \infty}T_n = \lim \sup_{n \to \infty} a_n$$
        as we wanted.

        Therefore joining the results we have that
        $$\inf_n a_n \leq \lim \inf_{n \to \infty} a_n \leq \lim \sup_{n \to \infty} a_n \leq \sup_n a_n$$
    \end{proof}
    \begin{proof}{\textbf{23}}
        We know that $(a_n)$ converges to some $a \in \R$ so let $\epsilon >0$ then
        $|a_n - a |<\epsilon$ when $n \geq N$ then we have that
        $$a-\epsilon<a_n<a+\epsilon$$
        but this also means that
        $$a-\epsilon\leq t_n\leq T_n \leq a+\epsilon$$
        and therefore their limits should be between that interval too, then
        $$a-\epsilon \leq \liminf_{n \to \infty} a_n \leq \limsup_{n \to \infty} a_n  \leq a+\epsilon$$
        Therefore this means that $\lim\inf_{n \to \infty} a_n$ and
        $\lim\sup_{n \to \infty} a_n$ both converge to $a$ or what it's the same
        $$\lim\inf_{n \to \infty} a_n = \lim\sup_{n \to \infty} a_n = \lim_{n \to \infty} a_n$$
    \end{proof}
    \begin{proof}{\textbf{24}}
        We know that 
        $\limsup\limits_{n \to \infty} a_n = \inf\{\sup\{a_n, a_{n+1}, ...\}\}$
        so this means that 
        $$\limsup\limits_{n \to \infty} -a_n = \inf\{\sup\{-a_n, -a_{n+1}, ...\}\}$$
        But since $\sup -A = -\inf A$ we have that
        $$\limsup\limits_{n \to \infty} -a_n = \inf\{-\inf\{a_n, a_{n+1}, ...\}\}$$
        We also know that $\inf -A = -\sup A$ therefore
        $$\limsup\limits_{n \to \infty} -a_n = -\sup\{\inf\{a_n, a_{n+1}, ...\}\}$$
        It follows then by definition that
        $$\limsup\limits_{n \to \infty} -a_n = -\liminf\limits_{n \to \infty} a_n$$
    \end{proof}
\cleardoublepage
    \begin{proof}{\textbf{25}}        
        We know that
        $\limsup_{n \to \infty} a_n = \lim_{n \to \infty} \sup_{k \geq n} a_k = -\infty$
        so this means that if we have an $M < 0$ then we can find an $N \in \N$
        such that if $k \geq N$ then $\sup_{k \geq N} a_k < M$ and in particular
        $a_k < M$ for any $k \geq N$. Therefore $\lim_{n \to \infty} a_n = -\infty$.

        Now we have that $\limsup_{n \to \infty} a_n = +\infty$ which means that \\
        $\lim_{n \to \infty}\sup_{k\geq n} a_k = +\infty$ so we have that for every $n$
        that $\sup_{k\geq n} a_k = +\infty$ in particular
        $$\sup_{k\geq 1} a_k > 1$$
        This means that exists $k_1 \geq 1$ such that $a_{k_1} > 1$. Now using
        again this fact we have that
        $$\sup_{k\geq k_1 + 1} a_k > 2$$ 
        Then we can find $k_2$ such that $k_2 > k_1$ and $a_{k_2} > 2$. Continuing this
        procedure we may find a set of increasing indices
        $$k_1 < k_2 < ... < k_n < k_{n+1} < ...$$
        such that $a_{k_n} > n$ therefore we have an unbounded increasing subsequence
        $(a_{k_n})$ that diverge to $+\infty$.

        Now let's see what happens when
        $\liminf_{n \to \infty} a_n = \lim_{n \to \infty}\inf_{k\geq N} a_n = +\infty$
        this means that if we have an $M>0$ then we can find an $N \in \N$ such that
        if $k\geq N$ then $\inf_{k\geq N} a_n > M$ and in particular $a_k > M$ for any
        $k \geq N$. Therefore $\lim_{n \to \infty} a_n = +\infty$.

        In the same way as above if $\liminf_{n \to \infty} a_n = -\infty$ this means
        that $\lim_{n \to \infty}\inf_{k\geq n} a_k = -\infty$ so we have that for every
        $n$ that $\inf_{k\geq n} a_k = -\infty$ in particular
        $$\inf_{k\geq 1} a_k < -1$$
        This means that exists $k_1 \geq 1$ such that $a_{k_1} < -1$. Now using
        again this fact we have that
        $$\inf_{k\geq k_1 + 1} a_k < -2$$ 
        Then we can find $k_2$ such that $k_2 > k_1$ and $a_{k_2} < -2$. Continuing this
        procedure we may find a set of increasing indices
        $$k_1 < k_2 < ... < k_n < k_{n+1} < ...$$
        such that $a_{k_n} < -n$ therefore we have an unbounded decreasing subsequence
        $(a_{k_n})$ that diverge to $-\infty$.

    \end{proof}
\cleardoublepage
    \begin{proof}{\textbf{26}}
        ($\rightarrow$)  We know that $M = \limsup_{n \to \infty} a_n$ which means that
        for some $\epsilon>0$ we can find $N \in \N$ such that if $n\geq N$ then
        $$|\sup_{k \geq n}a_k - M|< \epsilon$$
        or in other words
        $$M - \epsilon<\sup_{k \geq n}a_k < M+\epsilon$$
        We also know that $a_n \leq \sup_{k \geq n}a_k < M+\epsilon$ so
        $a_n < M+\epsilon$ but this only works for $n\geq N$ then this is
        only valid for all but finitely many $n<N$.
        
        Now let us suppose that $M-\epsilon > a_n$  we want to show by contradiction
        that this cannot be true for all $n \geq N$ then in that case the
        $\sup_{k \geq n} a_k \leq M -\epsilon$ but we saw by the definition of the
        $\limsup$ that $\sup_{k \geq n} a_k > M - \epsilon$ so we have a
        contradiction and if  $M-\epsilon > a_n$ happen then it cannot happen for all
        $n\geq N$ therefore $M-\epsilon<a_n$ happen for infinitely many $n$.

        ($\leftarrow$) Now we know that $M$ satisfies that for every $\epsilon >0$,
        we have $a_n < M+\epsilon$  for all but finitely many $n$, and $M-\epsilon<a_n$
        for infinitely many $n$.
        Since $(a_n)$ is bounded $M+\epsilon$ is an upper bound for $(a_n)$ and since
        this is true for all but finitely many $n$, let us take $N \in \N$ such that if
        $n \geq N$ we have that  $a_n < M + \epsilon$ then given that $(a_n)$ is bounded
        $(a_n)$ has a supremum and $\sup_{k\geq n} a_k \leq M+\epsilon$.

        Also since $M-\epsilon<a_n$ for infinitely many $n$, then
        $M - \epsilon < \sup_{k\geq n} a_k$ is true.
        Therefore $M = \limsup_{n\to \infty} a_n$\\
        
        Let us now caracterize $m = \liminf_{n \to \infty} a_n$ as
        \begin{equation*}
            \begin{cases}
                \text{for every $\epsilon>0$ we have $a_n<m+\epsilon$ for infinitely many $n$}\\
                \text{and $m-\epsilon < a_n$ for all but finitely many $n$}
            \end{cases}
        \end{equation*}

        ($\rightarrow$)  We know that $m = \liminf_{n \to \infty} a_n$ which means that
        for some $\epsilon>0$ we can find $N \in \N$ such that if $n\geq N$ then
        $$|\inf_{k \geq n}a_k - m|< \epsilon$$
        or in other words
        $$m - \epsilon<\inf_{k \geq n}a_k < m+\epsilon$$
        We also know that $m-\epsilon < \inf_{k \geq n} a_k \leq a_n$ so
        $m -\epsilon < a_n$ but this only works for $n\geq N$ then this is
        only valid for all but finitely many $n<N$.
        
        Now let us suppose that $a_n > m+\epsilon$  we want to show by contradiction
        that this cannot be true for all $n \geq N$ then if that were true the
        $\inf_{k \geq n} a_k \geq m +\epsilon$ but we saw by the definition of the
        $\liminf$ that $\inf_{k \geq n} a_k < m+ \epsilon$ so we have a
        contradiction and if  $a_n > m+\epsilon$ happen then it cannot happen for all
        $n\geq N$ therefore $a_n<m+\epsilon$ happen for infinitely many $n$.

        ($\leftarrow$) Now we know that $m$ satisfies that for every $\epsilon >0$,
        we have $a_n < m+\epsilon$  for infinitely many $n$, and $m-\epsilon<a_n$
        for all but finitely many $n$.
        Since $(a_n)$ is bounded $m-\epsilon$ is a lower bound for $(a_n)$ and since
        this is true for all but finitely many $n$, let us take $N \in \N$ such that if
        $n \geq N$ we have that  $m- \epsilon< a_n$ then given that $(a_n)$ is bounded
        $(a_n)$ has a infimum and $m-\epsilon \leq \inf_{k\geq n} a_k$.

        Also since $a_n<m+\epsilon$ for infinitely many $n$, then 
        $\inf_{k\geq n} a_k < m - \epsilon$ is true.
        Therefore $m = \liminf_{n\to \infty} a_n$\\
    \end{proof}
    \begin{proof}{\textbf{27}}
        The case where $M = \limsup_{n \to \infty} a_n = \pm\infty$ was handled in
        Excercise 25 so we will focus on the case where
        $M = \limsup_{n \to \infty} a_n \neq \pm\infty$.

        Let $\epsilon = 1$. Since $M = \limsup_{n \to \infty} a_n$ is caracterized by
        $(*)$ then there is some $N \in \N$ such that if $n \geq N$ then $a_n < M + 1$
        but we also have that $M - 1 < a_n$ for infinitely many $n$, so we can choose
        $n_1 \in \N$ such that $n_1 \geq N$ where both inequalities are satisfied.

        Similarly we can choose $n_2 > n_1 \geq N$ such that
        $a_{n_2} < M + \frac{1}{2}$ and $M + \frac{1}{2}< a_{n_2}$ for inifinitely many
        $n$. Then following this procedure we can find a subsequence $(a_{n_k})$ such
        that $|a_{n_k} - M| < \frac{1}{k}$ which implies that $(a_{n_k})$ converge to
        $M$.

        Let us now show that there is also a subsequence that converge to
        $\liminf_{n \to \infty} a_n$. We saw that is true in Excercise 25 for the case
        where  $\liminf_{n \to \infty} a_n = \pm \infty$.
        
        In the case that $m = \liminf_{n \to \infty} a_n \neq \pm\infty$ since 
        $\liminf$ is caracterized by an analogous statement given for $\limsup$ then
        the proof follows the same structure as above.
    \end{proof}
    \begin{proof}{\textbf{30}}
        If $a_n \leq b_n$ then $\inf a_n \leq \inf b_n$ and so
        $$\inf_{k \geq n}a_k \leq \inf_{k\geq n} b_k$$
        by applying the limit then we have that 
        $$\lim_{n\to\infty}\inf_{k \geq n} a_n \leq \lim_{n\to\infty}\inf_{k\geq n} b_k$$
        but we know that $(a_n)$ converge then because of Excercise 23 we have that
        $\lim_{n\to\infty} a_n = \liminf_{n\to\infty} a_n$, therefore
        $$\lim_{n\to\infty} a_n \leq \lim_{n\to\infty}\inf_{k\geq n} b_k$$
    \end{proof}
\cleardoublepage
    \begin{proof}{\textbf{32}}
        Let us prove first the case where $\limsup_{n\to\infty} a_n = +\infty$ then
        because of Exercise 25 we have that $(a_n)$ has a subsequence $(a_{k_n})$
        that diverge to $+\infty$ then 
        $$\sup S = \lim_{n\to\infty} a_{k_n} = \limsup_{n\to\infty} a_n = +\infty$$
        
        Now in the case where $\limsup_{n\to\infty} a_n = -\infty$
        because of Exercise 25 we have that $(a_n)$ diverge to $-\infty$ too so 
        every subsequence must diverge to $-\infty$ then
        $$\sup S = \limsup_{n\to\infty} a_n = -\infty$$
        
        Finally, suppose $\limsup_{n\to\infty} a_n \neq \pm\infty$ if we take any subsequence
        $(a_{k_n})$ we know that $a_{k_n} \leq \sup a_{k_n} \leq \sup a_n$  and then
        $$a_{k_n} \leq \sup_{m\geq n} a_m$$
        so if we apply the limit we have that
        $$\lim_{n \to \infty} a_{k_n} \leq \lim_{n \to \infty} \sup_{m\geq n} a_n$$
        Which implies that the limit of any subsequence of $(a_n)$ is less or equal to
        $\limsup_{n \to \infty} a_n$ therefore $\sup S = \limsup_{n \to \infty} a_n$.

        Let's see what happens with the infimum.
        Let us prove first the case where $\liminf_{n\to\infty} a_n = -\infty$ then
        because of Exercise 25 we have that $(a_n)$ has a subsequence $(a_{k_n})$
        that diverge to $-\infty$ then 
        $$\inf S = \lim_{n\to\infty} a_{k_n} = \liminf_{n\to\infty} a_n = -\infty$$
        
        Now in the case where $\liminf_{n\to\infty} a_n = +\infty$
        because of Exercise 25 we have that $(a_n)$ diverge to $+\infty$ too so 
        every subsequence must diverge to $+\infty$ then
        $$\inf S = \liminf_{n\to\infty} a_n = +\infty$$
        
        Finally, suppose $\liminf_{n\to\infty} a_n \neq \pm\infty$ if we take any subsequence
        $(a_{k_n})$ we know that $\inf a_n \leq \inf a_{k_n} \leq a_{k_n}$  and then
        $$\inf_{m\geq n} a_m \leq a_{k_n}$$
        so if we apply the limit we have that
        $$\lim_{n \to \infty} \inf_{m\geq n} a_m \leq \lim_{n \to \infty} a_{k_n}$$
        Which implies that the limit of any subsequence of $(a_n)$ is bigger or equal
        to $\liminf_{n \to \infty} a_n$ therefore $\inf S = \liminf_{n \to \infty} a_n$.
    \end{proof}
\cleardoublepage
    \begin{proof}{\textbf{40}}
    \begin{itemize}
        \item [$(i)\rightarrow(ii)$] We know that $\lim_{x \to a}f(x) = L$ so by the
        $\epsilon-\delta$ definition we have that for every $\epsilon>0$ there is some
        $\delta >0$ such that $|f(x)- L|<\epsilon$  whenever $x$ satisfies
        $0<|x-a|<\delta$.
        
        But also we know that $\lim_{n \to \infty} x_n = a$ then by the definition of
        the sequence limits we know that we can find some $N \in \N$ such that if $n\geq N$
        then $|x_n - a| < \epsilon'$ for any $\epsilon' > 0$ so if we take
        $\epsilon' = \delta$ then we have that $0<|x_n - a|<\delta$ because we also
        know that $x_n \neq a$. Therefore it must happen that $|f(x_n)-L|<\epsilon$
        for every $\epsilon >0$.

        \item [$(ii)\rightarrow(iii)$] 
        Let $y_n = f(x_n)$. From part $(i)\rightarrow(ii)$ we know we can find
        $N \in \N$ so if $n \geq N$ then $0<|x_n - a|<\delta$ and because of that 
        $|y_n - L|<\epsilon$ therefore $y_n$ converges to some limit $L$.

        \item [$(iii)\rightarrow(ii)$] If $(f(x_n))$ converges to something when
        $x_n\rightarrow a$ then there is some $N\in\N$ and when $n\geq N$ we have that
        $0<|x_n - a|<\epsilon'$    
        but this also means that $|f(x_n) - L|<\epsilon$ when
        $n\geq N$ because $(f(x_n))$ converges to something, but this means that there
        exists some number $L$ such that $f(x_n)\rightarrow L$ whenever
        $x_n \rightarrow a$ where $x_n \neq a$.
        
        So we know that $f(x_n)\rightarrow L$ when $x_n \rightarrow a$ but we want
        to check that this is true for any sequence we choose. Suppose we take another
        sequence $(y_n)$ and then we build $(z_n)$ as $z_n = x_1, y_1, x_2, y_2, ...$.
        Now let us suppose that $f(z_n) \rightarrow M$ then $f(y_n)$ and $f(x_n)$ are
        subsequences of $f(z_n)$ but we know that if a sequence converge to some number
        then all its subsequences must converge to the same number so
        $f(x_n) \rightarrow M$ and $f(y_n) \rightarrow M$ as we wanted.

        \item [$(ii)\rightarrow(i)$] Let us suppose that $\lim_{x \to a}f(x) \neq L$ we
        want to arrive to a contradiction. Then for some $\epsilon>0$ there isn't a 
        $\delta >0$ such that $|f(x)- L|<\epsilon$  whenever $x$ satisfies
        $0<|x-a|<\delta$.

        But from part $(iii)\rightarrow(ii)$ we know that when $0<|x_n-a|<\epsilon'=\delta$
        then $f(x_n) \rightarrow L$ so $|f(x_n)-L|<\epsilon$ therefore we found a set 
        of values of $x$ such that if $0<|x_n-a|<\epsilon'=\delta$
        then $|f(x_n)-L|<\epsilon$ so we have a contradiction and
        $\lim_{x \to a}f(x) = L$ must be true.
    \end{itemize}
    \end{proof}
\cleardoublepage
    \begin{proof}{\textbf{44}}
        Since $f$ is increasing and bounded then we can construct a sequence $f(n)$
        where $n \in \N$  such that $(f(n))$ is convergent because it's increasing and
        bounded. This means that there is $N \in \N$ such that when $n \geq N$ then
        $|f(n)-L|<\epsilon$ moreover since $(f(n))$ is bounded and increasing we can
        write that $f(n)\leq L$. Let $x\geq N$ then since $f$ is increasing we can write
        that $f(n) \leq f(x)$ therefore $|f(x)-L|<\epsilon$ for any $x \geq N$.

        In the case of $lim_{x \to -\infty}f(x)$ we can take a sequence $(f(-n))$ which
        is going to be a decreasing and bounded sequence and therefore convergent.
        This means that there is $N \in \N$ such that when $n \geq N$ then
        $|f(-n)-L'|<\epsilon$ moreover since $(f(-n))$ is bounded and decreasing
        we can write that $f(-n) \geq L'$. Let $x\leq -N$ since $f$ is increasing we
        can write that $f(-n) \geq f(x)$ therefore $|f(x)-L|<\epsilon$ for any
        $x \leq -N$.
    \end{proof}
    \begin{proof}{\textbf{45}}
        Suppose $c \in [a,b]$ and $c \notin \Q$ such that $f(c) \neq 0$. We want to arrive to
        a contradiction where we show that $f(c)$ must be equal to $0$.
        Since $f$ is continuous at $c$ let $\epsilon = |f(c)|/2$ then there is some
        $\delta>0$ such that when  $|x-c|<\delta$ we have that 
        $|f(x) - f(c)|<\epsilon$ so if $x$ is rational the expression becomes
        $|-f(c)|<\epsilon$  but we said that $\epsilon = |f(c)|/2$ and
        $|-f(c)| \nless |f(c)|/2$ therefore we have a contradiction and must be true
        that $f(c) = 0$.
    \end{proof}
    \begin{proof}{\textbf{46}}
    \begin{itemize}
        \item [(a)] Since $f(0)>0$ we can find an $\epsilon>0$ such that
        $f(0)-\epsilon>0$ (for example $\epsilon = \frac{f(0)}{2}$). Now from the
        definition of continuity let us choose $\epsilon>0$ such that $f(0)-\epsilon>0$
        then there is some $\delta>0$ such that when $|x-0|< \delta$ we have that
        $|f(x)-f(0)|<\epsilon$ then $f(0)-\epsilon<f(x)<f(0)+\epsilon$ so
        $0<f(0)-\epsilon<f(x)$.
        \item [(b)] Let $a \in \R$ and $a \notin \Q$ and let us suppose that $f(a)<0$
        we want to arrive at a contradiction where we show this cannot be true.
        We know from the analogous of part (a) that there is some interval, let's say
        $(a-\delta,a+\delta)$ where if $x \in (a-\delta,a+\delta)$ then $f(x)<0$ but
        also in this interval there is some rational $b$ for which $f(b)\geq 0$ then
        we have a contradiction and therefore must be true that $f(a)\geq 0$.
        
        This result doesn't hold if we change from $\geq 0$ to $>0$. Take for example
        $f(x)=|\pi-x|$ then $f(x)>0$ for all rationals but if $x=\pi$ then $f(\pi) = 0$. 
    \end{itemize}
    \end{proof}

\end{document}






















