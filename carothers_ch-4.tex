\documentclass[11pt]{article}
\usepackage{amssymb}
\usepackage{amsthm}
\usepackage{enumitem}
\usepackage{amsmath}
\usepackage{bm}
\usepackage{adjustbox}
\usepackage{mathrsfs}
\usepackage{graphicx}
\usepackage{siunitx}
\usepackage[mathscr]{euscript}

\title{\textbf{Solved selected problems of Real Analysis - Carothers}}
\author{Franco Zacco}
\date{}

\addtolength{\topmargin}{-3cm}
\addtolength{\textheight}{3cm}

\newcommand{\N}{\mathbb{N}}
\newcommand{\Z}{\mathbb{Z}}
\newcommand{\Q}{\mathbb{Q}}
\newcommand{\R}{\mathbb{R}}
\newcommand{\diam}{\text{diam}}
\newcommand{\cl}{\text{cl}}
\newcommand{\bdry}{\text{bdry}}
\newcommand{\inter}{\text{int}}

\theoremstyle{definition}
\newtheorem*{solution*}{Solution}

\begin{document}
\maketitle
\thispagestyle{empty}

\section*{Chapter 4 - Open Sets and Closed Sets}

	\begin{proof}{\textbf{1}}
        Let $U = (a,b)\times (c,d)$ and let $(x,y) \in U$ we want to check that
        if $(x',y') \in B_{\epsilon}((x,y))$ then $(x', y') \in U$. Since
        $(x',y') \in B_{\epsilon}((x,y))$ we have that  
        \begin{align*}
            d_\infty((x,y),(x',y')) = \max\{d(x,x'), d(y,y')\} &< \epsilon
        \end{align*}
        Then if $\max\{d(x,x'), d(y,y')\} = d(x,x')$ this means that\\
        $d(y,y') \leq d(x,x') < \epsilon$. Then $x' \in B_{\epsilon}(x)$ and since
        $(a,b)$ is an open set in $\R$ this means that $x' \in (a,b)$, the same can be
        shown for $y'$ such that $y' \in (c,d)$. Therefore $(x', y') \in U$. 

        Generalizing, let $U = A \times B$ and let $(a,b) \in U$ we want to check that
        if $(a',b') \in B_{\epsilon}((a,b))$ then $(a', b') \in U$. So in the same way
        since $(a',b') \in B_{\epsilon}((a,b))$ we have that  
        \begin{align*}
            d_\infty((a,b),(a',b')) = \max\{d(a,a'), d(b,b')\} &< \epsilon
        \end{align*}
        Then if $\max\{d(a,a'), d(b,b')\} = d(a,a')$ this means that\\
        $d(b,b') \leq d(a,a') < \epsilon$. Then $a' \in B_{\epsilon}(a)$ and since
        $A$ is an open set in $\R$ this means that $a' \in A$, the same can be
        shown for $b'$ such that $b' \in B$. Therefore $(a', b') \in U$.

        Let now $U = A \times B$ where $A$ and $B$ are closed sets in $\R$, we want to
        prove that $U$ is also closed in $\R^2$. We see that
        $(\R \setminus A) \times \R$ and $\R \times (\R \setminus B)$ are open sets because
        $\R \setminus A$, $\R \setminus B$ and $\R$ are open sets. Also, we know that
        the union of open sets is also an open set so
        $\R \times \R \setminus A \times B$ is also an open set which means that
        $A \times B$ must be a closed set.
    \end{proof}
\cleardoublepage
	\begin{proof}{\textbf{3}}
        \begin{itemize}
        \item [($\rightarrow$)] Let $x \in U$ where $U$ is an open set of $(M, d)$
        and let $(x_n)$ be a sequence that converges to $x$ since it is an open set we
        know that $x_n \in U$ for all but finitely many $n$, i.e. there is $N \in \N$
        such that for all $n \geq N$ we have that $d(x_n, x) < \epsilon$ for some
        $\epsilon > 0$ which means that $d(x_n, x) \to 0$ but since the metric $d$ and
        $\rho$ are equivalent then if $d(x_n, x) \to 0$ we have that $\rho(x_n, x) \to 0$.
        Therefore either $\rho$ or $d$ generate the same set $U$.
        \item [($\leftarrow$)] Let $U$ be an open set that is generated either by $d$ and by
        $\rho$ then if $x \in U$ and we have a sequence $(x_n)$ that converge to $x$ we
        know that $x_n \in U$ for all but finitely many $n$ i.e. there is $N \in \N$
        such that for all $n \geq N$ we have that $d(x_n, x) < \epsilon$ and
        $\rho(x_n, x) < \epsilon$ because they both generate $U$ this means that
        $d(x_n, x) \to 0$ and that $\rho(x_n, x) \to 0$ which implies that they are
        equivalent.
    \end{itemize}
    \end{proof}
	\begin{proof}{\textbf{6}}
        An example of an infinite closed set in $\R$ containing only irrationals is
        the set of all the square roots of the prime numbers, i.e.
        $$ F= \{\sqrt{2}, \sqrt{3}, \sqrt{5}, ... \sqrt{p_n},\sqrt{p_{n+1}}, ...\}$$
        So the complement of this set is the set
        $$F^c = (-\infty, \sqrt{2})~\cup~(\sqrt{2}, \sqrt{3})~\cup~...~\cup~
        (\sqrt{p_n},\sqrt{p_{n+1}})~\cup~...$$
        Where the intervals are open intervals of $\R$ and they are open sets
        plus the union of open sets is open, therefore $F^c$ is open and $F$ by
        definition is closed.

        Let us suppose that we have a set $F \subset \R$ that is an open set consisting
        entirely of irrationals we want to arrive at a contradiction. Let us grab an
        element $x \in F$ where by definition is irrational, then the ball around
        $x$ is defined as  $B_\epsilon(x) = (x-\epsilon, x+\epsilon)$, but we know that
        $\Q$ is a dense set in $\R$ so there is an element $q \in \Q$ such that
        $q \in (x-\epsilon, x+\epsilon)$ so we have a contradiction and
        $B_\epsilon(x) \not\subset F$. Therefore there is no open set consisting
        entirely of irrationals.
    \end{proof}
\cleardoublepage
	\begin{proof}{\textbf{7}}
        Let $F$ be an open set in $\R$ then for each $x \in F$ we have that there is
        $B_\epsilon(x) = (x-\epsilon, x+\epsilon)$ where $B_\epsilon(x) \subset F$. Also,
        we know that $\Q$ is dense in $\R$ so there is $a_x,b_x \in \Q$
        such that $a_x \in (x-\epsilon, x)$ and $b_x \in (x, x + \epsilon)$ so we have
        that $x \in (a_x, b_x)$ then we can write that
        $$F = \bigcup_{x \in F} (a_x,b_x)$$
        Finally, we see that each interval is in $\Q \times \Q$ and we know that
        $\Q \times \Q$ is equivalent to $\N$, therefore since the intervals involved are
        a subset of $\Q \times \Q$ they are countable.
        
        From what we proved we see that each open set $F$ is a countable union of
        intervals with rational endpoints, this suggests an injective function that
        sends an open set $F$ to  $F\cap\Q$ where $F \cap \Q \in \mathcal{P}(\Q)$
        so we can construct an injective map
        $$f: \mathcal{U} \to \mathcal{P}(\Q)$$
        Also notice that $\mathcal{P}(\Q)$ is equivalent to $\R$, so we can construct an
        injective map that sends $x \in \R$ to $(-\infty, x) \in\mathcal{U}$ i.e.
        we have a map $g: \mathcal{P}(\Q) \to \mathcal{U}$,
        therefore using Bernstein's Theorem we get that there is a bijective map
        $h:\mathcal{U} \to \mathcal{P}(\Q)$ implying that
        $$\text{card}(\mathcal{U}) = \text{card}(\R)$$
    \end{proof}
	\begin{proof}{\textbf{11}}
        Let $(x_n)$ be a sequence of sequences from $E = \{e^{(k)}: k\geq 1\}$ then
        $d(x_n, x_m) = 2$ if $x_n \neq x_m$ and $d(x_n, x_m) = 0 $ if $x_n = x_m$. This
        means that $(x_n)$ converges to some $x \in l_1$ if eventually $x = x_n$ but
        then $x \in E$. Therefore this implies that $E$ is a closed set of $l_1$ 
    \end{proof}
	\begin{proof}{\textbf{13}}
        Let $(x^{(n)})$ be a sequence of sequences from $c_0$ that converge to
        $x \in l_\infty$ we want to prove that also $x \in c_0$.
        Since $(x^{(n)})$ converges to $x \in l_\infty$ then for all $\epsilon > 0$ 
        there is $N \in \N$ such that when $n \geq N$ we have that
        $\|x^{(n)} - x\|_\infty < \epsilon$.
        Then we have that
        $$\|x^{(n)} - x\|_\infty = \sup_{k \in \N}|x^{(n)}_k - x_k|< \epsilon$$
        So we get that
        \begin{align*}
            |x_k| &= |x_k - x_k^{(n)} + x_k^{(n)}|\\
            &\leq |x_k - x_k^{(n)}| + |x_k^{(n)}|\\
            &\leq \sup_{k \in \N}|x^{(n)}_k - x_k| + |x^{(n)}_k|\\
            &< \epsilon + |x^{(n)}_k|
        \end{align*}
        And since $x^{(n)} \in c_0$ then $|x^{(n)}_k| \to 0$ when $k \to \infty$.
        Therefore $|x_k| < \epsilon$ implying that $|x_k| \to 0$ and that $x \in c_0$. 
    \end{proof}
	\begin{proof}{\textbf{15}}
        Let $A = \{y \in M: d(x,y) \leq r\}$ be the closed ball around $x$, we want
        to show that $M \setminus A$ is an open set which implies that $A$ is
        a closed set.
        If $M \setminus A$ is an open set then for every $z \in M \setminus A$
        there is an open ball $B_t(z)$ such that $B_t(z) \subset M \setminus A$.
        
        We have that $d(z,x) > r$ which implies that $d(z,x) - r > 0$ so let us
        define $t = d(z,x) - r$ then we have found $t > 0$ such that
        $B_t(z) \subset M \setminus A$ as we wanted.
        Therefore $B_t(z)$ is an open ball and $M \setminus A$ is
        an open set, which implies that $A$ is a closed set.

        Now let's see that $A$ is not necessarily equal to the closure of the open ball
        $B_r(x)$. Let's define a metric
        $$d(x,y) = \begin{cases}
            \text{$0 \quad$ if $x = y$}\\
            \text{$1 \quad$ otherwise}
        \end{cases}$$
        Then the open ball $B_{1}(x)$ with this metric is given by
        $$B_1(x) = \{y \in M: d(x,y) < 1\} = \{x\}$$
        So now we claim that $\cl(B_1(x)) = \{x\}$ we see this is true because let
        $y\in M$ such that $y \neq x$ so with this metric $d(x,y) = 1$ then there is an
        open ball $B_{1/2}(y) \subset M \setminus B_1(x)$ implying that $M \setminus B_1(x)$
        is open and $\{x\}$ is closed. Which is different from the closed ball
        $$A = \{y \in M: d(x,y) \leq 1\} = M$$
    \end{proof}
	\begin{proof}{\textbf{16}}
        Let $A = \{x \in V: \|x\| < 1\}$ and $B = \{x \in V: \|x\| \leq 1\}$. We
        know $x \in \bar{A}$ if there is a sequence $(x_n) \subset A$ such that
        $x_n \to x$. Then suppose $x \in V$ such that $\|x\| = 1$ and let us define a
        sequence $(x_n)$ that converge to $x$ as
        $$x_n = \frac{n-1}{n}x$$
        We see that $\|\frac{n-1}{n}x\| = |\frac{n-1}{n}|\|x\| = |\frac{n-1}{n}|\cdot 1 < 1$
        then $(x_n) \subset A$ and this implies that $x \in \bar{A}$. Therefore
        $B$ is always the closure of $A$.
    \end{proof}
	\begin{proof}{\textbf{17}}
        \begin{itemize}
        \item [($\rightarrow$)] If $A$ is an open set, since $\mathring{A}$ is the
        largest open set contained in $A$ then $\mathring{A} = A$.
        \item [($\leftarrow$)] If $\mathring{A} = A$ since $\mathring{A}$ is an open set
        then $A$ is open.
        \end{itemize}
        \begin{itemize}
        \item [($\rightarrow$)] If $A$ is closed, since $\bar{A}$ is the smallest closed
        set that contains $A$ then $\bar{A} = A$.
        \item [($\leftarrow$)] If $\bar{A} = A$ since $\bar{A}$ is a closed set
        then $A$ is closed.
        \end{itemize}
    \end{proof}
	\begin{proof}{\textbf{18}}
        Since $E$ is a nonempty bounded subset of $\R$ then there is a non-decreasing
        sequence $(x_n) \subset E$ where $\lim_{n \to \infty} x_n = \sup{E}$ therefore
        $\sup{E} \in \bar{E}$. In the same way, we know there is a non-increasing
        sequence $(x_n) \subset E$ where $\lim_{n \to \infty} x_n = \inf{E}$ therefore
        $\inf{E} \in \bar{E}$.
    \end{proof}
	\begin{proof}{\textbf{20}}
        Since $A \subset B$ and $B \subset \bar{B}$ then $A \subset \bar{B}$.
        Now let
        $$C = \{F: F\text{ is closed set and }A \subset F\}$$
        We know that
        $$\bar{A} = \bigcap\{F: F\text{ is closed set and }A \subset F\} = \bigcap C$$
        Then this means that $\bar{B} \in C$.
        Therefore since $\bar{A}$ is the intersection of $C$ we see that
        $\bar{A} \subset \bar{B}$.

        Let us now see why $\bar{A} \subset \bar{B}$ does not imply that
        $A \subset B$ by checking the following example. Let us
        define $A = (0, 1]$ and $B = \Q$ then $\bar{A} = [0, 1]$ and
        $\bar{B} = \R$ so we see that $\bar{A} \subset \bar{B}$ but
        $A \not\subset B$.
    \end{proof}
	\begin{proof}{\textbf{24}}
        Let $A \subset M$ so $A^c = M \setminus A$. Let us also define
        $$U = \bigcup\{F: F \text{ is open and } F\subset M\setminus A\} = \text{int}(A^c)$$
        So by definition, $U$ is an open set then $U^c = M \setminus U$ is
        closed and $A \subset U^c$ because of the definition of $U$
        also we see that $U^c$ must be the smallest closed set containing $A$
        again because of how we defined $U$. Therefore
        $$\text{cl}(A) = (\text{int}(A^c))^c$$

        Let us now define
        $$I = \bigcap\{F: F \text{ is closed and } M\setminus A \subset F\} = \text{cl}(A^c)$$
        So we see that $I$ is a closed set then $I^c = M \setminus I$ is
        open and $I^c \subseteq A$ because of the definition of $I$.
        Also, we see that $I^c$ must be the largest open set contained in $A$
        because of how we defined $I$. Therefore
        $$\text{int}(A) = (\text{cl}(A^c))^c$$
    \end{proof}
\cleardoublepage
    \begin{proof}{\textbf{26}}
        \begin{itemize}
            \item [($\rightarrow$)] Let $d(x, A) = 0$ then this means that
            $\inf\{d(x,a): a \in A\} = 0$ for which we have two cases. If $x \in A$
            then we have that 
            $$\min\{d(x,a): a \in A\} = \inf\{d(x,a): a \in A\} = d(x,x) = 0$$
            and we have that $x \in \bar{A}$ since $A \subset \bar{A}$.
            
            If $x \not\in A$ and we know that $\inf\{d(x,a): a \in A\} = 0$
            then it is possible to form a sequence $(x_n) \subset A$ such that
            $x_n \to x$ i.e. $d(x_n, x) \to 0$ which implie that
            $x \in \bar{A}$.
            
            \item [($\leftarrow$)] If $x \in \bar{A}$ then there is a sequence
            $(x_n) \subset A$ such that $x_n \to x$ which implies that
            $d(x, x_n) \to 0$ and since by definition of the metrics
            $d(x,a) \geq 0$ for any
            $a \in A$ then $\inf\{d(x,a): a \in A\} = 0$. Therefore
            $d(x, A) = 0$.
        \end{itemize}
    \end{proof}
    \begin{proof}{\textbf{28}}
        Let $D = \{x \in M: d(x,A) < \epsilon\}$ and let us define
        $\epsilon' = \epsilon - d(x,A)$ where we see that $\epsilon' > 0$.
        We want to prove that $B_{\epsilon'}(x) \subset D$ where we know that
        $B_{\epsilon'}(x) = \{y \in M:d(y,x) < \epsilon'\}$ then we have that
        \begin{align*}
            d(y,x) &< \epsilon - d(x,A)\\
            d(y,A) &\leq d(y,x) + d(x,A) < \epsilon
        \end{align*}
        Then this implies that $B_{\epsilon'}(x) \subset D$ and therefore $D$ is
        an open set.

        Let now $F = \{x \in M: d(x,A) \leq \epsilon\}$ and let us suppose that
        there is a sequence $(x_n) \subset F$ such that $x_n \to x$ where
        $x \in M$ then this implies that there is
        $N\in \N$ such that when $n \geq N$ we have that
        $d(x_n, x) < \epsilon'$ for some $\epsilon' > 0$. Also from problem 27
        we have that
        \begin{align*}
            |d(x, A) - d(x_n, A)| \leq d(x_n, x) < \epsilon'
        \end{align*}
        And from the triangle inequality, we see that
        \begin{align*}
            d(x, A) &= |d(x, A) - d(x_n, A) + d(x_n, A)| \leq\\
                &\leq|d(x, A) - d(x_n, A)| + |d(x_n, A)|
        \end{align*}
        Then
        \begin{align*}
            d(x,A) \leq \epsilon' + \epsilon
        \end{align*}
        In particular, let us take $\epsilon' = (d(x,A) - \epsilon)/2$ then
        we have that
        \begin{align*}
            d(x,A) &\leq \frac{d(x,A)}{2} + \frac{\epsilon}{2}\\
            d(x,A) &\leq \epsilon
        \end{align*}
        Therefore $x \in F$ which implies that $F$ is a closed set.

        Finally, if $x \in A$ we have that $d(x,A) = d(x,x) = 0 < \epsilon$
        which implies that $A \subset D$ and $A \subset F$.
    \end{proof}
\cleardoublepage
    \begin{proof}{\textbf{29}}
        \item[(i)]
        From the hint we have, we see that each set\\
        $\{x \in M : d(x,A) < 1/n\}$ is an open set. Let's see that
        $$\bigcap_{n=1}^{\infty}\{x \in M : d(x,A) < 1/n\} = \{x \in M : d(x,A) = 0\}$$
        So, we need to prove that $d(x,A) = 0$ if
        and only if for all $n$ it holds that $d(x,A) < 1/n$.
        \begin{itemize}
        \item [($\rightarrow$)] If $d(x,A) = 0$ then $d(x, A) = 0 < 1/n$ for all
        $n \in \N$.
        \item [($\leftarrow$)] On the other hand, if $d(x,A) < 1/n$ for all $n$
        then let us suppose $d(x,A) > 0$ we want to arrive to a contradiction.
        We know that there is $n \in \N$ such that $n > 1/d(x,A)$ then
        $d(x,A) > 1/n$ which is a contradiction. Therefore it must be that
        $d(x,A) = 0$.
        \end{itemize}
        Also we know that $d(x,A) = 0$ if and only if $x \in \bar{A}$ so we have
        that
        $$\{x \in M : d(x,A) = 0\} = \bar{A}$$
        Where we know that $\bar{A}$ is closed.
        Therefore every closed set in $M$ is the intersection of countably many
        open sets.

        \item[(ii)] Now let's see that $\{x \in M : d(x,A) \geq 1/n\}$ is the
        complement of the set $\{x \in M : d(x,A) < 1/n\}$ which implies that
        $\{x \in M : d(x,A) \geq 1/n\}$ is a closed set. Then because of 
        what we saw in part (i) we have that
        $$\bigcup_{n=1}^{\infty}\{x \in M : d(x,A) \geq 1/n\} = \bar{A}^c$$
        And we know that $\bar{A}^c$ is open, therefore every open set in $M$ is
        the intersection of countably many closed sets.
    \end{proof}
    \begin{proof}{\textbf{33}}
        Let $(B_\epsilon(x) \setminus \{x\}) \cap A = \{x_1, x_2, ..., x_n\}$
        i.e. $B_\epsilon(x) \setminus \{x\}$ has finitely many points of $A$
        for all $\epsilon > 0$ we want to arrive to a contradiction.
        
        Let us take $x_m \in \{x_1, x_2, ..., x_n\}$ such that
        $$d(x_m, x) = \min\{d(x_1, x), d(x_2, x), ..., d(x_n, x)\}$$

        Since $(B_\epsilon(x) \setminus \{x\}) \cap A \neq \emptyset$ 
        for all $\epsilon > 0$ in particular let us take $\epsilon' = d(x,x_m)$
        then we see that $(B_{\epsilon'}(x) \setminus \{x\}) \cap A = \emptyset$
        which is a contradiction and therefore
        $(B_\epsilon(x) \setminus \{x\}) \cap A$ hast infinitely many points.
    \end{proof}
\cleardoublepage
    \begin{proof}{\textbf{34}}
        \begin{itemize}
        \item [($\rightarrow$)] Let $x$ be a limit point of $A$ then let us
        construct a sequence $(x_n) \subset ((B_\epsilon(x) \setminus \{x\}) \cap A)$
        which we know it exists because 
        $(B_\epsilon(x) \setminus \{x\}) \cap A \neq \emptyset$. We can
        construct the sequence by taking
        $x_n \in (B_{1/n}(x) \setminus \{x\}) \cap A)$.
        We see that for each $x_n$ we have that $0 < d(x_n, x) < 1/n$.
        Therefore $x_n \to x$ and by definition of $(x_n)$ we know that
        $(x_n) \subset A$ and $x_n \neq x$ for all $n$.
        \item [($\leftarrow$)] Let $(x_n) \subset A$ such that $x_n \to x$ and
        $x_n \neq x$ for all $n$. Then this implies that given some $\epsilon > 0$
        for $n \geq N$ we have that $d(x_n,x) < \epsilon$ where $N \in \N$.
        So we have that $x_n \in ((B_\epsilon(x) \setminus \{x\}) \cap A)$
        for all $n \geq N$ by the definition of an open ball. Therefore since
        $\epsilon$ is arbitrary we have that
        $B_\epsilon(x) \setminus \{x\}) \cap A \neq \emptyset$ i.e. $x$ is a
        limit point of $A$.
        \end{itemize}
    \end{proof}
    \begin{proof}{\textbf{41}}
        \begin{itemize}
        \item [(a)] Let $x \in \bdry(A)$ then we know that
        $$B_{\epsilon}(x) \cap A \neq \emptyset \quad\text{and}\quad
        B_{\epsilon}(x) \cap A^c \neq \emptyset$$ 
        for every $\epsilon > 0$. And since $(A^c)^c = A$ we also have that
        $$B_{\epsilon}(x) \cap (A^c)^c \neq \emptyset$$
        This also implies that $x \in \bdry(A^c)$. In the same way, we can show
        that if $x \in \bdry(A^c)$ then also $x \in \bdry(A)$. And since $x$ is
        arbitrary we get that $\bdry(A) = \bdry(A^c)$.
        
        \item [(b)] Let $x \in \bdry(A)$ so we know that
        $B_{\epsilon}(x) \cap A \neq \emptyset$ and that
        $B_{\epsilon}(x) \cap A^c \neq \emptyset$
        for every $\epsilon > 0$. This implies that $x \in \cl(A)$ too then
        we have that
        $$\bdry(A) \subset \cl(A)$$
        We also have that $\inter(A) \subseteq A \subset \cl(A)$ therefore
        we have that
        $$(\bdry(A) \cup \inter(A)) \subset \cl(A)$$

        Now, let $x \in \cl(A)$ and also it could happen that
        $x \in \inter(A)$ or $x \not\in \inter(A)$.
        We are interested in the case where $x \not\in \inter(A)$. Suppose
        $x \not\in \bdry(A)$ we want to arrive to a contradiction then
        $B_{\epsilon}(x) \cap A = \emptyset$ or
        $B_{\epsilon}(x) \cap A^c = \emptyset$. If
        $B_{\epsilon}(x) \cap A = \emptyset$ then this implies that
        $x \not\in \cl(A)$ because of Proposition 4.10 which is a contradiction.
        So it must be that $B_{\epsilon}(x) \cap A^c = \emptyset$ then
        $B_{\epsilon}(x) \subset A$ but since $B_{\epsilon}(x)$ is an open set
        this implies that $B_{\epsilon}(x) \subset \inter(A)$ which is another
        contradiction to the fact that $x \not\in \inter(A)$. Then
        it must be that $x \in \bdry(A)$ and
        $$\cl(A) \subset (\bdry(A) \cup \inter(A))$$
        Therefore
        $$\cl(A) = (\bdry(A) \cup \inter(A))$$
        \item [(c)] Let $x \in M$ we want to prove that then
        $x \in \inter(A) \cup \bdry(A) \cup \inter(A^c)$. Suppose that
        $x \in \bdry(A)$ then by definition $x \in M$ so let us suppose that
        $x \not\in \bdry(A)$ then $B_{\epsilon}(x) \cap A = \emptyset$ or
        $B_{\epsilon}(x) \cap A^c = \emptyset$.
        
        If $B_{\epsilon}(x) \cap A = \emptyset$ then $B_{\epsilon}(x) \subset A^c$
        and since $B_{\epsilon}(x)$ is an open set this implies that
        $B_{\epsilon}(x) \subset \inter(A^c)$.
        
        And if $B_{\epsilon}(x) \cap A^c = \emptyset$ then
        $B_{\epsilon}(x) \subset A$ and since $B_{\epsilon}(x)$ is an open set
        this implies that $B_{\epsilon}(x) \subset \inter(A^c)$.
        
        Then in the first case $x \in \inter(A^c)$ and in the second case
        $x \in \inter(A)$ therefore
        $$M = \inter(A) \cup \bdry(A) \cup \inter(A^c)$$
    \end{itemize}
    \end{proof}
    \begin{proof}{\textbf{42}}
        Let $E$ be a nonempty bounded subset of $\R$ and let $M = \sup E$ then
        we see that $B_{\epsilon}(M) = (M - \epsilon, M + \epsilon)$ and
        by definition we know that $x \leq \sup E$ for all $x \in E$ then
        there is $y \in E$ such that $M - \epsilon < y \leq M$
        so it must happen that $B_{\epsilon}(M) \cap E \neq \emptyset$. Also, we
        see that $M < M+\epsilon$ so $M+\epsilon \not\in E$ then it must happen that
        $M+\epsilon \in E^c$ then $B_{\epsilon}(M) \cap E^c \neq \emptyset$.
        Therefore since $\epsilon$ was arbitrary we have that
        $\sup E \in \bdry(E)$ as we wanted.

        In the same way let $m = \inf E$
        we see that $B_{\epsilon}(m) = (m - \epsilon, m + \epsilon)$ and
        by definition we know that $\inf E \leq x$ for all $x \in E$ then
        there is $y \in E$ such that $m \leq y < m + \epsilon$ so it must happen
        that $B_{\epsilon}(m) \cap E \neq \emptyset$. Also, we see that
        $m - \epsilon < m$ so $m - \epsilon \not\in E$ then it must happen that
        $m - \epsilon \in E^c$ then $B_{\epsilon}(m) \cap E^c \neq \emptyset$.
        Therefore since this is valid for all $\epsilon > 0$ we have that
        $\inf E \in \bdry(E)$ as we wanted.
    \end{proof}
\cleardoublepage
    \begin{proof}{\textbf{43}}
        We want to prove that $\bdry(A)$ is a closed set.
        But first, we will
        check that $\inter(A)$, $\inter(A^c)$ and $\bdry(A)$ are mutually
        disjoint.

        Since $\inter(A) \subseteq A$ and $\inter(A^c) \subseteq A^c$ and also
        $A$ is disjoint from $A^c$ we have that $\inter(A)$ is disjoint from
        $\inter(A^c)$.

        Let $x \in \inter(A)$ then there is a ball
        $B_{\epsilon}(x) \subset \inter(A)$ for some $\epsilon > 0$. Let us also
        suppose that $x \in \bdry(A)$ we want to arrive to a contradiction then
        by definition we know that  $B_{\epsilon'}(x) \cap A \neq \emptyset$ and
        $B_{\epsilon'}(x) \cap A^c \neq \emptyset$ for every $\epsilon' > 0$ but
        we showed that there is some $\epsilon > 0$ such that
        $B_{\epsilon}(x) \subset \inter(A)$
        then $B_{\epsilon}(x) \cap A^c = \emptyset$ which is a contradiction.
        Therefore $\inter(A)$ is disjoint from $\bdry(A)$.
        
        Finally, let $x \in \inter(A^c)$ then there is
        a ball $B_{\epsilon}(x) \subset \inter(A^c)$ for some $\epsilon > 0$.
        Let us also suppose that $x \in \bdry(A)$ we want to arrive to a
        contradiction then by definition we know that
        $B_{\epsilon'}(x) \cap A \neq \emptyset$ and
        $B_{\epsilon'}(x) \cap A^c \neq \emptyset$ for every $\epsilon' > 0$ but
        we showed that there is some $\epsilon > 0$ such that
        $B_{\epsilon}(x) \subset \inter(A^c)$
        then $B_{\epsilon}(x) \cap A = \emptyset$ which is a contradiction.
        Therefore $\inter(A^c)$ is disjoint from $\bdry(A)$.

        Now let's prove that $\bdry(A)$ is a closed set. From the problem 41(c)
        we have that $M = \inter(A) \cup \bdry(A) \cup \inter(A^c)$ and since we
        proved that $\inter(A)$, $\inter(A^c)$ and $\bdry(A)$ are mutually
        disjoint we have that $M \setminus \bdry(A) = \inter(A) \cup \inter(A^c)$
        and since $\inter(A)$, $\inter(A^c)$ and the union of open sets is open
        then $M \setminus \bdry(A)$ is an open set. Therefore $\bdry(A)$ is
        a closed set.

        Finally, we want to prove that $\bdry(A) = \cl(A) \setminus \inter(A)$
        then from the problem 41(b) we have that
        $\cl(A) = \inter(A) \cup \bdry(A)$ also since $\inter(A)$ and $\bdry(A)$
        are mutually disjoint  we have that
        $\cl(A) \setminus \inter(A) = \bdry(A)$
    \end{proof}
\cleardoublepage
    \begin{proof}{\textbf{46}}
    \begin{itemize}
    \item [(a)] ($\rightarrow$)
    If $A$ is dense in $M$ then $\bar{A} = M$ also we know that if $x \in \bar{A}$
    i.e. $x \in M$ then there is a sequence $(x_n) \subset A$ such that
    $x_n \to x$. So every point in $M$ is the limit of a sequence from $A$.
    
    ($\leftarrow$) If every point in $M$ is the limit of a sequence from $A$
    then there is a sequence $(x_n) \subset A$ such that $x_n \to x$ also, we
    know that this implies $x \in \bar{A}$. Therefore $\bar{A} = M$ and $A$ is
    dense in $M$.

    \item [(b)] ($\rightarrow$) If $A$ is dense in $M$ then $\bar{A} = M$ so
    if $x \in \bar{A}$ i.e. $x \in M$ we know from Proposition 4.10 that
    $B_{\epsilon}(x) \cap A \neq \emptyset$ for every $\epsilon > 0$ as we
    wanted.

    ($\leftarrow$) If $B_{\epsilon}(x) \cap A \neq \emptyset$ for every $x \in M$
    and every $\epsilon > 0$ again from Proposition 4.10 we have that $x$ must
    be in $\bar{A}$. Therefore $\bar{A} = M$ and $A$ is dense in $M$.

    \item [(c)] ($\rightarrow$) If $A$ is dense in $M$ then $\bar{A} = M$. Let
    $x \in U$ also we have that  $x \in \bar{A}$ then $x \in \inter(A)$ or
    $x \in \bdry(A)$ since they are disjoint. If $x \in \inter(A)$ then $x \in A$
    and $U \cap A \neq \emptyset$.
    If $x \in \bdry(A)$ since $U$ is an open neighborhood of $x$ there is some
    $\epsilon > 0$ for which $B_{\epsilon}(x) \subseteq U$ and by definition
    of boundary $B_{\epsilon}(x) \cap A \neq \emptyset$. Therefore
    $U \cap A \neq \emptyset$ for every nonempty open set $U$.

    ($\leftarrow$) Let us suppose $\bar{A} \neq M$ and let $x \not\in \bar{A}$
    (so $x \in \bar{A}^c \neq \emptyset$)
    then there is an $\epsilon >0$ for which an open ball 
    $B_{\epsilon}(x) \cap  A = \emptyset$ but
    we know that $U \cap A \neq \emptyset$ for every nonempty open set $U$
    so this cannot happen since
    $B_{\epsilon}(x)$ is a nonempty open set. Therefore it must happen that
    $\bar{A} = M$.

    \item [(d)] ($\rightarrow$) Suppose $x \in \inter(A^c)$ i.e.
    $\inter(A^c) \neq \emptyset$ we want to arrive to a contradiction. We
    know by part (c) that $U \cap A \neq \emptyset$ for every nonempty open set
    $U$, but $\inter(A^c)$ is open by definition of interior so
    $\inter(A^c) \cap A \neq \emptyset$ which is a contradiction. Therefore it
    must happen that $\inter(A^c) = \emptyset$.

    ($\leftarrow$) Let $\inter(A^c) = \emptyset$ we want to show that
    $\bar{A} = M$. We know that $\bar{A} = (\inter(A^c))^c$ therefore we have
    that $\bar{A} = \emptyset^c = M$. 
    \end{itemize}        
    \end{proof}
    \begin{proof}{\textbf{48}}
        An example of countable dense set in $\R$ is $\Q$, in the same way
        for $\R^2$ we can take $\Q \times \Q$ since a cartesian product of
        countable sets is also countable and for $\R^n$ we take $\Q^n$.
    \end{proof}
\cleardoublepage
    \begin{proof}{\textbf{51}}
        Let $M$ be a separable metric space and let $C$ be the set that contains
        all the isolated points of $M$. Let $x \in C$ i.e. $x$ is an isolated
        point of $M$ then we have that $B_{\epsilon}(x) \cap M = \{x\}$
        for some $\epsilon > 0$. Since $M$ is a separable metric space
        then there is a countable dense subset $A$ such that $M = \bar{A}$.
        So also, $x \in \bar{A}$ which means that either $x \in \inter(A)$ or
        $x \in \bdry(A)$. If $x \in \bdry(A)$ then it must happen that
        $B_{\epsilon'}(x) \cap A \neq \emptyset$ for every $\epsilon' > 0$ but we
        showed that $B_{\epsilon}(x) \cap M = \{x\}$ for some $\epsilon>0$ 
        so it must happen that $x \in A$. And if $x \in \inter(A)$ then
        $x \in A$. Then in any case we have that $x \in A$ which means that
        $C \subseteq A$ but $A$ is a countable set, therefore $C$ must be
        countable too.
    \end{proof}
    \begin{proof}{\textbf{52}}
        Let $M$ be a separable metric space and let $W$ be a collection of
        disjoint open sets in $M$. Let $U \in W$ where $U$ is an open set.
        Since $M$ is a separable metric space then there is a countable dense
        subset $A$ such that $M = \bar{A}$ and we showed that for any nonempty
        open set $U$ it must happen that $U \cap A \neq \emptyset$ then we can
        build a map $f:W \to A$ such that for any $U \in W$ we assign a value
        $x \in (U \cap A)$. Finally, we need to check that this mapping
        is one to one, suppose there is $U, V \in W$ such that $f(U) = f(V)$
        then $x \in U \cap A$ and $x \in V \cap A$ but
        we know that $U$ and $V$ are disjoint sets then it must happen that
        $U = V$.
        Therefore since $A$ is countable and we have a map between every element
        in $W$ to a value $x \in A$ then $W$ should be at most countable. 
    \end{proof}
\cleardoublepage
    \begin{proof}{\textbf{61}}
        We are asked to prove (ii) and (iii) from Proposition 4.13.
        \begin{itemize}
        \item [(ii)] We want to prove that a set $F \subset A$ is closed in
        $(A,d)$ if and only if $F = A \cap C$ where $C$ is closed in $(M,d)$.

        ($\rightarrow$) Suppose $F$ is closed in $(A,d)$ and let us suppose that
        $C = \cl_{M}(F)$ then we see that $F \subseteq A \cap C$. Now let
        $x \in A \cap C$ then $x \in A$ and $x \in \cl_{M}(F)$ so there is a
        sequence $(x_n) \subset F$ such that $x_n \to x$ where $x \in (M,d)$ but
        also since $F$ is a closed set it must happen that $x\in F$ i.e.
        $A \cap C \subseteq F$.
        Therefore joining both inclusions we have that $F = A \cap C$.
        
        ($\leftarrow$) Suppose $F = A \cap C$ where $C$ is closed in $(M,d)$.
        Let $(x_n) \subset F$ such that $x_n \to x$ and $x \in A$ we want to prove
        that $x \in F$. Since $(x_n) \subset F$ then $(x_n) \subset C$ and since
        $C$ is a closed set then it must happen that $x \in C$ then
        $x \in F$. Therefore $F$ is closed in $(A,d)$.
        
        \item [(iii)] We want to prove that $\cl_A(E) = A \cap \cl_M(E)$ for any
        subset $E$ of $A$.

        Let $x \in \cl_A(E)$ then $B_{\epsilon}^A(x) \cap E \neq \emptyset$ for
        every $\epsilon > 0$. But we also know that
        $B_{\epsilon}^A(x) = A \cap B_{\epsilon}^M(x)$ then we have that
        $A \cap B_{\epsilon}^M(x) \cap E \neq \emptyset$
        but since $E \subset A$ then $A \cap E = E$ so we have that
        $B_{\epsilon}^M(x) \cap E \neq \emptyset$ for every $\epsilon > 0$
        which implies that $x \in \cl_M(E)$ then
        $\cl_A(E) \subseteq A \cap \cl_M(E)$.

        Let $x \in A \cap \cl_M(E)$ then since $x \in \cl_M(E)$ we have that
        $B_\epsilon^M(x) \cap E \neq \emptyset$ for every $\epsilon > 0$.
        Since also $x \in A$ and $E \subset A$ then
        $B_\epsilon^M(x) \cap A \cap E \neq \emptyset$ hence
        $B_\epsilon^A(x) \cap E \neq \emptyset$ for every $\epsilon > 0$.
        Therefore $x \in \cl_A(E)$ i.e. $A \cap \cl_M(E) \subseteq \cl_A(E)$.
        
        Finally, by joining both inclusions we have that
        $\cl_A(E) = A \cap \cl_M(E)$.
        \end{itemize}
    \end{proof}

\cleardoublepage
    \begin{proof}{\textbf{62}}

        ($\rightarrow$) Let $G$ be open in $A$ then by Proposition 4.13 (i) we
        know that there
        is an open set $U$ in $M$ such that $G = A \cap U$. So let $x \in G$,
        then $x \in A$ and $A$ is open in $M$ so there is $\epsilon_A > 0$
        such that $B_{\epsilon_A}^M(x) \subset A$. Also, $x \in U$ and $U$
        is open in $M$ so there is $\epsilon_U > 0$ such that
        $B_{\epsilon_U}^M(x) \subset U$. Let us take
        $\epsilon = \min(\epsilon_A, \epsilon_U)$ so we have a ball
        $B_\epsilon^M(x) \subset A$ and $B_\epsilon^M(x) \subset U$ but also
        since $A \subset M$ and $U \subset M$ it happens that
        $B_\epsilon^M(x) \subset M$. Therefore $G$ is open in $M$.

        ($\leftarrow$) Let $G$ be open in $M$ and let $x \in G$ then there is a
        ball $B_{\epsilon}^M(x) \subset M$ for some $\epsilon > 0$. Also, we
        know that $B_{\epsilon}^A(x) = A \cap B_{\epsilon}^M(x)$ which is
        nonempty since $G \subset A$. Therefore we have a ball
        $B_{\epsilon}^A(x) \subset A$ for $\epsilon > 0$ i.e. $G$ is open in
        $A$.

        Now let us prove that if $A$ is closed in $(M,d)$ and $G \subset A$
        then $G$ is closed in $A$ if and only if $G$ is closed in $M$.

        ($\rightarrow$) Let $G$ be closed in $A$ and let $x \in M$
        such that for any $\epsilon > 0$ we have that
        $B_{\epsilon}^M(x) \cap G \neq \emptyset$, we want to prove that
        $x \in G$. Since $G \subset A$ then $G \cap A = G$ so we can write that
        $B_{\epsilon}^M(x) \cap A \cap G \neq \emptyset$. Also, let us suppose
        that $x \not\in A$, we want to arrive to a contradiction then $x \in A^c$
        which is an open set because $A$ is a closed set then there is a ball
        $B_{\epsilon'}^M(x) \subset A^c$ for some $\epsilon' > 0$, but we said
        that $B_{\epsilon}^M(x) \cap A \cap G \neq \emptyset$ for any $\epsilon > 0$
        then we have a contradiction and $x \in A$. So since $x$ in $A$ we have
        that $B_{\epsilon}^A(x) = A \cap B_\epsilon^M(x)$ then we get that
        $B_{\epsilon}^A(x) \cap G \neq \emptyset$ for all $\epsilon > 0$
        therefore $x \in G$ because $G$ is closed in $A$ which implies that $G$
        is closed in $M$.
        
        ($\leftarrow$) Let $G$ be closed in $M$ and let $x \in A$
        such that for any $\epsilon > 0$ we have that
        $B_{\epsilon}^A(x) \cap G \neq \emptyset$, we want to prove that
        $x \in G$. Since $x$ in $A$ then we know that
        $B_{\epsilon}^A(x) = A \cap B_\epsilon^M(x)$ so we get that
        $A \cap B_{\epsilon}^M(x) \cap G \neq \emptyset$ also, since
        $G \subset A$ then $G \cap A = G$ then we have that
        $B_{\epsilon}^M(x) \cap G \neq \emptyset$ for any $\epsilon > 0$
        therefore $x \in G$ because $G$ is closed in $M$. Hence $G$ is closed 
        in $A$. 

    \end{proof}
    \begin{proof}{\textbf{64}}
        Let $E = A = \Q$ then we see that $\inter_A(E) = \inter_\Q(\Q) = \Q$ but
        $\inter_\R(E) = \inter_\R(\Q) = \emptyset$.
    \end{proof}
\cleardoublepage
    \begin{proof}{\textbf{69}}

        ($\rightarrow$) Let $(U_n)$ be a countable open base for $M$ and let us
        take an element from each $U_n$ such that for each $n \in \N$ we have
        $x_n \in U_n$, then we can construct a set $D = \{x_n : n \in \N\}$
        which is countable. Also, let $U$ be a nonempty open set of $M$ then
        it can be written as a union of $U_n$ which implies that
        $U \cap D \neq \emptyset$. Therefore $D$ is a countable dense set of $M$
        i.e. $M$ is separable.

        ($\leftarrow$) Let $M$ be a separable metric space, let $F$ be an
        open set in $M$ and let $\{x_n\}$ be a countable dense set of $M$.
        Let us take $x \in F$ then we have a ball
        $B_\epsilon(x) \subset F$ for some $\epsilon > 0$.
        Also, we can have $B_q(x) \subset B_\epsilon(x) \subset F$ such that
        $q \in \Q$ and $0 < q < \epsilon$ because $\Q$ is dense.
        Since $\{x_n\}$ is dense then there is some $x_n \in B_q(x)$
        and which implies that $x \in B_q(x_n)$ so we can write that
        $$F = \bigcup_n B_q(x_n)$$
        Therefore $F$ can be written as a union of countable open sets i.e. $M$
        has a countable open base. 
    \end{proof}

\end{document}






















