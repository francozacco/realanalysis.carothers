\documentclass[11pt]{article}
\usepackage{amssymb}
\usepackage{amsthm}
\usepackage{enumitem}
\usepackage{amsmath}
\usepackage{bm}
\usepackage{adjustbox}
\usepackage{mathrsfs}
\usepackage{graphicx}
\usepackage{siunitx}
\usepackage[mathscr]{euscript}

\title{\textbf{Solved selected problems of Real Analysis - Carothers}}
\author{Franco Zacco}
\date{}

\addtolength{\topmargin}{-3cm}
\addtolength{\textheight}{3cm}

\newcommand{\N}{\mathbb{N}}
\newcommand{\Z}{\mathbb{Z}}
\newcommand{\Q}{\mathbb{Q}}
\newcommand{\R}{\mathbb{R}}
\newcommand{\diam}{\text{diam}}
\newcommand{\cl}{\text{cl}}
\newcommand{\bdry}{\text{bdry}}
\newcommand{\inter}{\text{int}}

\theoremstyle{definition}
\newtheorem*{solution*}{Solution}

\begin{document}
\maketitle
\thispagestyle{empty}

\section*{Chapter 4 - Continuous Functions}

	\begin{proof}{\textbf{7}}
    \begin{itemize}
        \item [(a)] Let $(a, \infty) \subset \R$ which is an open set, then
        we see that
        $$f^{-1}[(a,\infty)] = \{x: x \in M \text{ and } f(x) > a\}$$
        is also an open set because $f$ is continuous and Theorem 5.1
        part (iv).

        In the same way, let $(-\infty, a) \subset \R$ which is an open set,
        then we see that
        $$f^{-1}[(-\infty,a)] = \{x: x \in M \text{ and } f(x) < a\}$$
        is also an open set because $f$ is continuous and Theorem 5.1
        part (iv).

        \item [(b)] We proved the more general result in part (c) which also
        applies in this case.

        \item [(c)] Let $V$ be an open set of $\R$ then since the collection of
        open intervals with rational endpoints is a base for $\R$ we can write
        $V$ as
        $$V = \bigcup_\alpha (p_\alpha, q_\alpha)$$
        where $p_\alpha, q_\alpha \in \Q$ so we have that
        $$f^{-1}[V] = \bigcup_\alpha f^{-1}[(p_\alpha, q_\alpha)]$$
        then we can write that
        $$f^{-1}[(p_\alpha, q_\alpha)] =
        f^{-1}[(p_\alpha, \infty)] \cap f^{-1}[(-\infty, q_\alpha)]$$
        Also, we know that
        $$f^{-1}[(p_\alpha, \infty)] = \{x: f(x) > p_\alpha\}
        \text{ and } f^{-1}[(-\infty, q_\alpha)] = \{x: f(x) < q_\alpha\}$$
         and we know
        both of them are open sets so $f^{-1}[(p_\alpha,q_\alpha)]$ is
        the intersection of a finite number of open sets then it is also an open
        set. Finally, since $f^{-1}[V]$ is the union of open sets it's also
        an open set. Therefore $f$ is continuous. 
    \end{itemize}
    \end{proof}
    \begin{proof}{\textbf{10}}
        Let $\epsilon > 0$ and let us take $\delta = 1$ no matter the value of
        $\epsilon$ then
        $$B_\delta(2) = \{x \in A: d(2,x) < 1\} = \{2\}$$
        So we have that $f(B_\delta(2)) = \{f(2)\}$ and certainly it must happen
        that $\{f(2)\} \subset B_\epsilon(f(2))$ because
        $f(2) \in B_\epsilon(f(2))$. Therefore $f$ is continuous at 2.
    \end{proof}
    \begin{proof}{\textbf{11}}
        \begin{itemize}
            \item [(a)] Let $x \in A \cup B$, then $x \in A$, $x \in B$ or both
            of them, also let $\epsilon > 0$ then we know there exists
            $\delta > 0$ such that $f(B_\delta(x)) \subset B_\epsilon(f(x))$
            because $f$ is continuous at $x$ by the definition.
            \item [(b)] Let $A = (0,1)$ and $B = [1,2)$ also let $f:A \to \R$ be
            defined as $f(x) = x$ and $f: B\to\R$ as $f(x) = x + 1$ then we
            see that $f:A \cup B \to \R$ is not continuous at $x=1$. Therefore
            the statement is false.            
        \end{itemize}
    \end{proof}
    \begin{proof}{\textbf{14}}
        Given that a continuous function on $\R$ is completely determined by its
        values on $\Q$. For each $q \in \Q$ we have that $f(q)$ has a
        cardinality of $\mathfrak{c}$ since for each real number $x \in \R$ we
        can find an $f$ such that $x = f(q)$. So the set of continuous functions
        $f: \R \to \R$ has a cardinality of $\mathfrak{c}^{|\Q|}$ and doing some
        cardinality algebra we get that
        $$\mathfrak{c}^{|\Q|} = {(2^{\aleph_0})}^{\aleph_0} = 2^{\aleph_0} = \mathfrak{c}$$
        Therefore there are $\mathfrak{c}$ continuous function $f: \R \to \R$. 
    \end{proof}
    \begin{proof}{\textbf{17}}
        Let $x \in M$, and let us also define $(x_n) \subset D$ such that
        $x_n \to x$ which we know it exists because $D$ is dense. Then
        $f(x_n) \to f(x)$ and $g(x_n) \to g(x)$ also we know that
        $f(x_n) = g(x_n)$ for every $x_n \in (x_n)$. Finally, since sequences
        have unique limits it must happen that $f(x) = g(x)$ as we wanted to
        show.

        In the same way, suppose we define $(x_n) \subset D$ such that
        $x_n \to x$ where $x \in M$. Then $f(x_n) \to f(x)$ also
        $(f(x_n)) \subset f(D)$. So we have a sequence $(f(x_n))$ for any
        $f(x)$ and we know that every $y \in N$ has the form $y = f(x)$
        because $f$ is onto. Therefore $f(D)$ is dense in $N$.
    \end{proof}
\cleardoublepage
    \begin{proof}{\textbf{22}}
        Let $n,m \in \N$ we want to show that $d(E(n), E(m)) = d(n,m)$ this
        means that $\|E(n) - E(m)\|_1 = |n-m|$. Let us suppose that $n > m$
        then $n = m + b$ where $b \in \N$ and so $|n - m| = b$.
        Then we have that
        \begin{align*}
            \|E(n) - E(m)\|_1 &= \sum_{i=1}^{\infty} |E_i(n) - E_i(m)|
        \end{align*}
        where $E_i(n)$ is the value of the $ith$ element in the sequence, the
        same for $E_i(m)$. If $i \in \{1, 2, ..., m \}$ we have that
        $|E_i(n) - E_i(m)| =  |1 - 1| = 0$ and for $i \in \{n +1, n+2, ...\}$
        we have that $|E_i(n) - E_i(m)| =  |0 - 0| = 0$ so we can write the
        following
        \begin{align*}
            \|E(n) - E(m)\|_1 &= \sum_{i=m}^{n} |E_i(n) - 0|
                = \sum_{i=m}^{n} 1 = n - m = b
        \end{align*}
        Therefore $\|E(n) - E(m)\|_1 = |n-m|$ as we wanted, in the case of
        $m \geq n$ the proof is analogous because we are taking the absolute
        value inside the sum.
    \end{proof}
    \begin{proof}{\textbf{23}}
        Let $S:c_0 \to c_0$ be defined as $S(x_0, x_1, ...) = (0, x_0, x_1, ...)$
        such that $S$ shifts the entries forward and puts $0$ in the empty slot.
        We want to prove that $d(S(x), S(y)) = d(x,y)$ where $x = (x_0, x_1, ...)$
        and $y = (y_0, y_1, ...)$ then
        $$\|x - y\|_\infty = \sup \{|x_0 - y_0|, |x_1 - y_1|, ...\}$$
        And $\sup \{|x_0 - y_0|, |x_1 - y_1|, ...\} \geq 0$ because both $x$
        and $y$ are sequences that tend to $0$. Let us suppose that
        $\sup_n |x_n - y_n| = 0$ then we see that
        $$\|S(x) - S(y)\|_\infty = \sup \{|0 - 0|, |x_0 - y_0|, |x_1 - y_1|, ...\}
        = 0$$
        Now let us suppose that $\|x - y\|_\infty > 0$ then we have that
        $$\sup \{|x_0 - y_0|, |x_1 - y_1|, ...\} = \sup \{|0-0|, |x_0 - y_0|, |x_1 - y_1|, ...\}$$
        because $\|S(x) - S(y)\|_\infty$ cannot be $0$. Therefore we have that
        $$\|S(x)-S(y)\|_\infty = \|x -y\|_\infty$$
        as we wanted.

    \end{proof}

\end{document}






















