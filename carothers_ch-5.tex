\documentclass[11pt]{article}
\usepackage{amssymb}
\usepackage{amsthm}
\usepackage{enumitem}
\usepackage{amsmath}
\usepackage{bm}
\usepackage{adjustbox}
\usepackage{mathrsfs}
\usepackage{graphicx}
\usepackage{siunitx}
\usepackage[mathscr]{euscript}

\title{\textbf{Solved selected problems of Real Analysis - Carothers}}
\author{Franco Zacco}
\date{}

\addtolength{\topmargin}{-3cm}
\addtolength{\textheight}{3cm}

\newcommand{\N}{\mathbb{N}}
\newcommand{\Z}{\mathbb{Z}}
\newcommand{\Q}{\mathbb{Q}}
\newcommand{\R}{\mathbb{R}}
\newcommand{\diam}{\text{diam}}
\newcommand{\cl}{\text{cl}}
\newcommand{\bdry}{\text{bdry}}
\newcommand{\inter}{\text{int}}

\theoremstyle{definition}
\newtheorem*{solution*}{Solution}

\begin{document}
\maketitle
\thispagestyle{empty}

\section*{Chapter 4 - Continuous Functions}

	\begin{proof}{\textbf{7}}
    \begin{itemize}
        \item [(a)] Let $(a, \infty) \subset \R$ which is an open set, then
        we see that
        $$f^{-1}[(a,\infty)] = \{x: x \in M \text{ and } f(x) > a\}$$
        is also an open set because $f$ is continuous and Theorem 5.1
        part (iv).

        In the same way, let $(-\infty, a) \subset \R$ which is an open set,
        then we see that
        $$f^{-1}[(-\infty,a)] = \{x: x \in M \text{ and } f(x) < a\}$$
        is also an open set because $f$ is continuous and Theorem 5.1
        part (iv).

        \item [(b)] We proved the more general result in part (c) which also
        applies in this case.

        \item [(c)] Let $V$ be an open set of $\R$ then since the collection of
        open intervals with rational endpoints is a base for $\R$ we can write
        $V$ as
        $$V = \bigcup_\alpha (p_\alpha, q_\alpha)$$
        where $p_\alpha, q_\alpha \in \Q$ so we have that
        $$f^{-1}[V] = \bigcup_\alpha f^{-1}[(p_\alpha, q_\alpha)]$$
        then we can write that
        $$f^{-1}[(p_\alpha, q_\alpha)] =
        f^{-1}[(p_\alpha, \infty)] \cap f^{-1}[(-\infty, q_\alpha)]$$
        Also, we know that
        $$f^{-1}[(p_\alpha, \infty)] = \{x: f(x) > p_\alpha\}
        \text{ and } f^{-1}[(-\infty, q_\alpha)] = \{x: f(x) < q_\alpha\}$$
         and we know
        both of them are open sets so $f^{-1}[(p_\alpha,q_\alpha)]$ is
        the intersection of a finite number of open sets then it is also an open
        set. Finally, since $f^{-1}[V]$ is the union of open sets it's also
        an open set. Therefore $f$ is continuous. 
    \end{itemize}
    \end{proof}
    \begin{proof}{\textbf{10}}
        Let $\epsilon > 0$ and let us take $\delta = 1$ no matter the value of
        $\epsilon$ then
        $$B_\delta(2) = \{x \in A: d(2,x) < 1\} = \{2\}$$
        So we have that $f(B_\delta(2)) = \{f(2)\}$ and certainly it must happen
        that $\{f(2)\} \subset B_\epsilon(f(2))$ because
        $f(2) \in B_\epsilon(f(2))$. Therefore $f$ is continuous at 2.
    \end{proof}
    \begin{proof}{\textbf{11}}
        \begin{itemize}
            \item [(a)] Let $x \in A \cup B$, then $x \in A$, $x \in B$ or both
            of them, also let $\epsilon > 0$ then we know there exists
            $\delta > 0$ such that $f(B_\delta(x)) \subset B_\epsilon(f(x))$
            because $f$ is continuous at $x$ by the definition.
            \item [(b)] Let $A = (0,1)$ and $B = [1,2)$ also let $f:A \to \R$ be
            defined as $f(x) = x$ and $f: B\to\R$ as $f(x) = x + 1$ then we
            see that $f:A \cup B \to \R$ is not continuous at $x=1$. Therefore
            the statement is false.            
        \end{itemize}
    \end{proof}
\end{document}






















